\input ucwmac2
\ucwmodule{verb}

\def\pair#1#2{$\left<#1,#2\right>$}
\def\E{{\bf E}}
\def\O{{\cal O}}
\def\frac#1#2{{{#1}\over{#2}}}
\let\f\frac
\mathchardef\comma="013B
% FIXME: This is ugly, no need to switch modes.
\def\eq{\ifmmode$\nobreak\ \discretionary{=}{=}{=}\nobreak\ $\else$\eq$\errmessage{You shall use \eq in the math mode.}\fi}
\def\cdotop{\mathop\cdot}
\def\degree{^{\circ}}
\def\nsd{\mathop{\rm nsd}\nolimits}
\def\nsn{\mathop{\rm nsn}\nolimits}

% Function typeset in italic, e.g., $\f{size}(v)$
\def\f#1{\mathop{\<#1>}\mkern 1mu\relax}

% Sets
\font\mft=msbm10
\def\N{\hbox{\mft N}}
\def\Z{\hbox{\mft Z}}
\def\Q{\hbox{\mft Q}}
\def\R{\hbox{\mft R}}
\def\C{\hbox{\mft C}}

% \vect is longer than \vec, but with smaller arrow than \overrightarrow (same arrow as in \vec)
\def\vect#1{%
  \vbox{\m@th \ialign {##\crcr
  \vectfill\crcr\noalign{\kern-\p@\vskip-2pt \nointerlineskip}
  $\hfil\displaystyle{#1}\hfil$\crcr}}}
\def\vectfill{%
  $\m@th\smash-\mkern-7mu%
  \cleaders\hbox{$\mkern-2mu\smash-\mkern-2mu$}\hfill\hskip-2.8pt
  \mkern-7mu\raise-3.7pt\hbox{$\mathord\mathchar"017E$}$}

% \<...> is already defined in ucwmac, but it does not scale the fonts in (super|sub)scripts
\def\<#1>{\ifmmode
\mathchoice{\hbox{\I #1}}{\hbox{\I #1}}{\hbox{\it\setfonts[/7]#1\/}}{\hbox{\it\setfonts[/5]#1\/}}
\else\leavevmode\hbox{\I #1}\fi}



\pdfpageheight=0cm
\pdfpagewidth=0cm

\output{\a}

\pdfhorigin 1pt
\pdfvorigin 1pt

\let\ex\expandafter
\ex\def\csname hex0\endcsname{\advance\count5 0\relax}
\ex\def\csname hex1\endcsname{\advance\count5 1\relax}
\ex\def\csname hex2\endcsname{\advance\count5 2\relax}
\ex\def\csname hex3\endcsname{\advance\count5 3\relax}
\ex\def\csname hex4\endcsname{\advance\count5 4\relax}
\ex\def\csname hex5\endcsname{\advance\count5 5\relax}
\ex\def\csname hex6\endcsname{\advance\count5 6\relax}
\ex\def\csname hex7\endcsname{\advance\count5 7\relax}
\ex\def\csname hex8\endcsname{\advance\count5 8\relax}
\ex\def\csname hex9\endcsname{\advance\count5 9\relax}
\ex\def\csname hexa\endcsname{\advance\count5 10\relax}
\ex\def\csname hexb\endcsname{\advance\count5 11\relax}
\ex\def\csname hexc\endcsname{\advance\count5 12\relax}
\ex\def\csname hexd\endcsname{\advance\count5 13\relax}
\ex\def\csname hexe\endcsname{\advance\count5 14\relax}
\ex\def\csname hexf\endcsname{\advance\count5 15\relax}
\def\printnum{
	\count6\count5
	\divide\count6 255
	\edef\tmp{\tmp\the\count6}
	\multiply\count6 -255
	\advance\count5 \count6
	\multiply\count5 10
}
\def\cset#1#2#3
{
	\count5=0
	\csname hex#2\endcsname
	\multiply\count5 16\relax
	\csname hex#3\endcsname
	\def\tmp{}\printnum\edef\tmp{\tmp.}
	\printnum\printnum
	\printnum\printnum
	\expandafter\edef\csname #1\endcsname,{\tmp}
}
\def\cdef#1#2#3#4#5#6
{
	\cset{cr}#1#2
	\cset{cg}#3#4
	\cset{cb}#5#6
}
\def\c{\colorlocal{\rgb{\cr, \cg, \cb,}}}
\cdef %%XPP_TEXT_COLOR%% 
00ff00

\newbox\b
\setbox\b=\hbox{
	\vrule height 0pt width 0pt
	\c
	%\verb|%%XPP_TOOL_INPUT%% |
	$
	%\vrule height 10pt width 10ptx
	\displaystyle
    %%XPP_TOOL_INPUT%%
% \cr,\ \cg,\ \cb,\
	$
}

\def\a{\shipout\vbox{\box\b}\setbox5\box255}

a

\bye

